\section{Diskussion}
\label{sec:Diskussion}
Der Literaturwert für die effektive Masse ist $m = 0.063m_e$ \cite{Masse}.
Für die bestimmten effektiven Massen ergeben sich relative Abweichungen von $99,86\%$ für $m_1$ und
$99,69\%$ für $m_2$. Für den Mittelwert der beiden Massen ergibt sich eine Abweichung von $99,78\%$.
Dies sind relativ hohe Abweichungen, dafür dass der Versuchsaufbau ordentlich eingestellt und justiert wurde und
die Messung überwiegend nach Plan ablief. Es sei aber angemerkt, dass die am Goniometer abgelesen Winkel
große Bereiche aufwiesen in denen das Oszilloskop konstant ein minimales Signal angezeigt hat. Dort wurde versucht
genau die Mitte des Winkelbereichs als Messergebnis aufzunehmen, um den Fehler möglichst gering zu halten.
Außerdem gab es bei der zweiten Probe zwei Winkel bei denen sich ein minimales Signal am Oszilloskop ablesen ließ.
Der Grund dafür konnte aber nicht bestimmt werden.
Weiterhin lässt sich fest stellen, dass die Messung des B-Feldes sehr genau funktioniert hat und deshalb als
Fehlerquelle ausgeschlossen werden kann. Aufgrund der hohen relativen Abweichungen konnte die effektive Masse
der Leitungselektronen in GaAs nicht zufriedenstellend bestimmt werden. Die Messung reicht aber aus, um
die Größenordnung der effektiven Masse ungenau an zu geben.
\newpage