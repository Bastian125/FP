\section{Diskussion}
\label{sec:Diskussion}

Die horizontale Komponente der Erdmagnetfeldes wurde mit $\qty{25.9(0.6)}{\micro\tesla}$ gemessen.
Dies entspricht in etwa der theoretischen Erwartung, die eine horizontale Feldstärke im Bereich um
$\qty{20}{\micro\tesla}$ vorhersagt. Die Abweichung ist hier höchswahrscheinlich darauf zurückzuführen,
dass der Tisch nicht vollständig parallel zum Magnetfeld ausgerichtet war, sowie dass die vertikale
Komponente möglicherweise auch nicht vollständig kompensiert wurde.

Die g-Faktoren sowie die Kernspins liegen sehr nah am theoretischen Wert. Für das Isotop
$^{87}\symup{Rb}$ sind die Messwerte unter Berücksichtigung der Fehlertoleranz sogar mit der
Theorie vereinbar.
Die Ergebnisse für $^{85}\symup{Rb}$ liegen selbst mit Fehlertoleranz leicht außerhalb der Theorie,
allerdings nur mit geringer Abweichung von unter $\qty{2}{\percent}$. Da die Unsicherheit beim Ablesen der Positionen der
Peaks nicht berücksichtigt wurde, ist es gut denkbar, dass die tatsächlichen Fehler im Endergebniss
größer sind und die Messwerte somit doch mit der Theorie übereinstimmen.

Unter Berücksichtigung der Anreicherung des $^{85}\symup{Rb}$ Isotops ist auch das zu $0.42$
berechnete Isotopenverhältnis als sinnvolles Ergebnis einzustufen.