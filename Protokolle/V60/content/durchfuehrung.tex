\section{Durchführung}
\label{sec:Durchführung}
Es werden drei Messungen mit unterschiedlichen Messaufbauten aufgenommen, um die Bedienung des Lasers zu üben.
Aus Sicherheitsgründen wurden Schutzbrillen getragen und, um die Messungen so störfrei wie möglich zu machen,
wurde der Raum teilweise abgedunkelt.

\subsection{Messen des Schwellenstrom}
\label{sec:Schwellenstrom}
Für den weiteren Versuch ist es wichtig den Schwellenstrom zu messen, da unter dem Schwellenstrom der Laser nur
als LED funktioniert und dementsprechend kein kohärents Licht erzeugt. Dafür wird der Laser an eine Apparatur
angeschlossen mit der der Strom variiert werden kann. Weiterhin wird eine Kamera vor eine Detektorkarte
positioniert, der Strom notiert bei dem Lasergranulation auftritt und entsprechende Fotos gemacht.
Die Detektorkarte wird so positioniert, dass der Strahl auf sie auftrifft. Lasergranulation (auch "Speckle" genannt)
ist ein Beugungseffekt, der durch die Unebenheit der Detektorkarte entsteht und ist leicht durch ein gepunktetes
Muster um die Auftreffstelle des Laserstrahls zu erkennen.

\subsection{Aufnahme der Rubidiumsfluoresenz}
\label{sec:Rubidium}
