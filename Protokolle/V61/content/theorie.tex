\section{Zielsetzung}
\label{sec:Zielsetzung}
In diesem Versuch soll die Funktion des Helium-Neon Lasers (HeNe-Laser) erprobt werden. Dazu wird eine
Stabilitätsmessung durchgeführt, zwei TEM-Moden und die Polarisation des Lasers gemessen. Außerdem wird noch die
Wellenlänge des Lasers mit Hilfe eines optischen Gitters bestimmt.

\section{Theorie}
\label{sec:Theorie}
\subsection{Grundaufbau eines Lasers}
\label{sec:Laser}
Ein Laser besteht im Allgemeinen aus einer Pumpquelle, einem aktiven Medium und einem optischen Resonator. Bei
einem HeNe-Laser wird für die Pumpquelle eine Spannung angelegt, so dass Elektronen das aktive Medium
anregen. Das aktive Medium ist eine Helium-Neon-Gemisch mit einem Verhältnis von 5:1. Die Elektronen regen
nun das Helium an, dass wiederum durch Stöße das Neon anregt. Da der angeregte Zustand im Neon wesentlich
langlebieger als im Helium ist, lässt sich dadurch die benötigte Besetzungsinversion herstellen. Bei
Besetzungsinversion sind alle Elektronen im angeregten Zustand. Ohne Pumpquelle wäre die Erwartung, dass
die Elektronen nach der Boltzmannstatistik verteilt wären und dadurch eher die energetisch niedrigeren
Zustände besetzen würden. Allgemein haben Elektronen nun zwei Möglichkeiten den energetisch höheren
Zustand wieder zu verlassen. Zum einem die spontane Emission wobei durch Quantenfluktuation das Elektron
in einen niedrigeren Zustand herab fällt und dabei ein Photon mit der Energiedifferenz der Zustände emittiert.
Zum anderen die stimulierte Emission, bei der ein Photon mit der Energie entsprechenden Energie einfällt und
das Elektron aus dem Zustand heraus löst. Dabei wird ein weiteres identisches Photon erzeugt. Um die gewünschte
Verstärkung des Lichts zu erhalten muss die induzierte gegenüber der sponaten Emission überwiegen. Dies ist
durch die Besetzungsinversion gegeben. Als Resonator werden zwei Spiegel verwendet. Einer von beiden ist hierbei
teildurchlässig um das Laserlicht als Strahl heraus zu lassen. Damit es zu Oszillatorverhalten kommt, müssen
die Reflektionsverluste der Spiegel möglichst gering gehalten werden. Je nach verwendeten Spiegel wird unter
planparallelen und sphärischen Resonatoren unterschieden. Wobei auch Kombinationen aus beiden möglich sind.
Außerdem befinden sich vor den Spiegeln noch Brewsterfenster, die nur p-polarisiertes Licht hindurchlassen und
das s-polarisierte Licht aus dem Strahl beugen. Diese werden verwendet um Reflektionsverluste zu vermeiden und
um die Verstärkung des Lichts nur linear zu polarisieren.

\subsection{Stabilität des Lasers}
\label{sec:Stabilitaet}
Die Stabilitätsbedingung für einen Laser ist durch
\begin{equation}
    \label{eq:stabil}
    0 \leq g_1 \cdot g_2 \leq 1
\end{equation}
gegeben. Dabei sind $g_1$ und $g_2$ die Resonatorparameter, die sich durch
\begin{equation}
    \label{eq:param}
    g_i = 1 - \frac{L}{r_i}
\end{equation}
mit der Resonatorlänge $L$ und den Krümmungsradien $r_i$ der Spiegel ergeben. Da die Wellenlänge des Lasers
bei $\lambda = \SI{632,8}{\nano\meter}$ liegt und dadurch wesentlich kleiner als die Resonatorlänge $L$ ist,
erfüllen viele Frequenzen die Resonatorbedingung.

\subsection{TEM-Modem}
\label{sec:Moden}
Die Anzahl der möglichen Wellenlängen werden als longitudinale Moden beschrieben. Da der Spiegel des Resonators
Unebenheiten aufweisen oder verkippt sein kann, kann es auch zu transversalen Moden kommen. Die Moden lassen sich
durch die Schreibweise $TEM_{lqp}$ beschreiben. Dabei gibt $q$ die longitudinale Mode an, $l$ und $p$ die $x$-
und $y$-Position der Knoten. Dabei ist die $TEM_{00}$-Mode die mit der höchsten Symmetrie und den geringsten
Verlusten. Allgemein lässt sich die Intensitätsverteilung durch eine Gaußverteilung und für die Mode entsprechenden
Hermite-Polynomen durch
\begin{equation}
    \label{eq:I}
    I_{lp}(x) \propto I_0 |H_{l}(x)H_{p}(x)e^{-\frac{x^2}{2}}|^2
\end{equation}
ausdrücken. Dies lässt sich wegen $I_{lp} \propto |E_{lp}|^2$ aus
\begin{equation}
    E_{lp}(x) \propto H_{l}(x)H_{p}(x)e^{-\frac{x^2}{2}}
\end{equation}
herleiten. Die Intensität der $TEM_{00}$-Mode ist dementsprechend nur eine Gaußgverteilung.

\subsection{Wellenlänge}
\label{sec:Wellenlaenge}
Wird durch ein optisches Gitter ein Interferenzmuster auf einem Schirm geworfen, kann aus den Abständen der
Maxima auf dem Schirm die Wellenlänge durch
\begin{equation}
    \label{eq:winkel}
    \lambda = \frac{g^{-1}}{n} \sin\left(\tan^{-1}\left(\frac{s_n}{d}\right)\right)
\end{equation}
bestimmen. Dabei ist $\lambda$ die Wellenlänge, $g^{-1}$ die Konstante des Gitters, $n$ ist die Orndung des
Maximums, $s_n$ der Abstand des n-ten Maximums zum Maximum 0. Ordnung und $d$ der Abstand des Gitters zum
Schirm.