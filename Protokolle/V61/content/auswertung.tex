\section{Auswertung}
\label{sec:Auswertung}
Im folgenden werden die Messergebnisse ausgewertet, dabei werden die 2 Spiegelkonfigurationen zuerst getrennt
voneinander betrachtet.

\subsection{Stabilitätsbedingung}
Zuerst wird die Stabilitätsbedingung überprüft.
Aus \eqref{eq:stabil} lässt sich ableiten, dass die maximale Resonatorlänge $L$ für eine
stabile Konfiguration gegeben ist durch
\begin{equation}
  \notag
  L_\text{max} = r_1 + r_2
\end{equation}
wobei $r_1$ der Krümmungsradius des ersten Spiegels und $r_2$ der Radius vom zweiten Spiegel ist.
Bei der ersten Spiegelkonfiguration ist $r_1=r_2=\qty{1400}{\milli\metre}$, die maximale Länge beträgt
somit $L_\text{max} = \qty{280}{\centi\metre}$. 
In \autoref{tab:L1} ist die Laserintensität bei verschiedenen Resonatorlängen aufgelistet.
\begin{table}[H]
  \centering
  \begin{tabular}{c|c}
    {$L$ / $\unit{\centi\metre}$} & {$I$ / $\unit{\milli\watt}$} \\
    \hline
    50  & 1,44  \\
    75  & 1,7  \\
    100  & 0,8  \\
    125  & 1,5  \\
    150  & 1,38  \\
    175  & 0,81   \\
    200  & 1,3 
  \end{tabular}
  \caption{Laserintensitäten in Abhängigkeit der Resonatorlänge für die erste Spiegelkonfiguration ($r_1=r_2=\qty{1400}{\milli\metre}$).}
  \label{tab:L1}
\end{table}
Die maximale Länge konnte dabei nicht erreicht werden, da die optische Schiene des Versuchsaufbaus zu kurz ist.
Innerhalb der theoretischen Grenze konnte allerdings ein stabiler Laserbetrieb nachgewiesen werden.
In der zweiten Spiegelkonfiguration wurde der zweite konkave Spiegel durch einen planen Spiegel ersetzt.
Die Formel für die maximale mögliche Resonatorlänge vereinfacht sich deshalb zu
\begin{equation}
  \notag
  L_\text{max} = r_1
\end{equation}
und liefert in diesem Fall $L_\text{max} = \qty{140}{\centi\metre}$.
Analog zur vorherigen Messung sind die Intensitäten in \autoref{tab:L2} festgehalten.
\begin{table}[H]
  \centering
  \begin{tabular}{c|c}
    {L / $\unit{\centi\metre}$} & {I / $\unit{\milli\watt}$} \\
    \hline
    50  & 1,4  \\
    75  & 1,83  \\
    100  & 1,87  \\
    125  & 2,7  \\
    135  & 2,6  \\
    138  & 1,8  \\
    140  & 2,09  \\
  \end{tabular}
  \caption{Laserintensitäten in Abhängigkeit der Resonatorlänge für die zweite Spiegelkonfiguration ($r_1=\qty{1400}{\milli\metre}$, $r_2 = \infty$).}
  \label{tab:L2}
\end{table}
Hier konnte bishin zu maximal theoretischen Resonatorlänge $L_\text{max} = \qty{140}{\centi\metre}$ ein Laserbetrieb nachgewiesen werden und
somit insgesamt auch die Gültigkeit der Stabilitätsbedingung aus \eqref{eq:stabil}.

\subsection{TEM Moden}
Um die transversalen TEM Moden zu verifizieren, werden die Messwerte der Intensität graphisch dargestellt und anschließend ein Fit an die theoretische
Funktion durchgeführt.
\subsubsection{$\text{TEM}_{00}$ Mode}
Zuerst wird die $\text{TEM}_{00}$ Mode analysiert. Die Messwerte sind in \autoref{fig:TEM00} geplottet.
Die theoretische Intensitätsverteilung in x-Richtung lautet gemäß \eqref{eq:I}
\begin{equation}
  \notag
  I_{\text{TEM}_{00}} = I_0 \exp \left( \frac{-\left(x - x_0\right)^2}{2\omega^2} \right)
\end{equation}
und wird mithilde der Python Erweiterung \textit{Scipy} \cite{scipy} an die Messwerte gefittet.
Die Fitparameter ergeben sich zu
\begin{align*}
  I_0 &=\qty{126+-5}{\micro\watt} \\
  x_0 &=\qty{0.2336+-0.0023}{\milli\metre}   \\
 \omega &=\qty{2.2373+-0.0023}{\milli\metre},
\end{align*}
der resultierende Fit ist ebenfalls in \autoref{fig:TEM00} dargestellt.
\begin{figure}[H]
  \centering
  \includegraphics{plot_TEM00.pdf}
  \caption{Messwerte sowie Fit der Intensitätsverteilung in x-Richtung für die $\text{TEM}_{00}$ Mode.}
  \label{fig:TEM00}
\end{figure}
\subsubsection{$\text{TEM}_{01}$ Mode}
Für die $\text{TEM}_{01}$ Mode wird analog verfahren, jedoch lautet die theoretische Intensitätsverteilung laut \eqref{eq:I} hier
\begin{equation}
  \notag
  I_{\text{TEM}_{01}} = I_1 \frac{8\left( x - x_0\right)^2}{\omega^2} \exp \left( \frac{-\left(x - x_0\right)^2}{2\omega^2} \right).
\end{equation}
Die Fitparameter lauten nun
\begin{align*}
  I_1 &=\qty{3.460+-0.016}{\micro\watt} \\
  x_0 &=\qty{1.377+-0.006}{\milli\metre}   \\
  x_1 &=\qty{-1.499+-0.006}{\milli\metre}   \\
 \omega &=\qty{2.2820+-0.0026}{\milli\metre},
\end{align*}
Die Messwerte sowie der Fit sind in \autoref{fig:TEM01} graphisch festgehalten.
\begin{figure}[H]
  \centering
  \includegraphics{plot_TEM01.pdf}
  \caption{Messwerte sowie Fit der Intensitätsverteilung in x-Richtung für die $\text{TEM}_{01}$ Mode.}
  \label{fig:TEM01}
\end{figure}

\subsection{Polarisationsmessung}
Auch für die Polarisation des Lasers soll mithilfe eines Fits die theoretische Erwartung geprüft werden.
In \autoref{fig:polarisation} sind die Messwerte der Intensität in Abhängigkeit des Winkels des Polarisationsfilters $\varphi$
geplottet.
Da der Laser in der Theorie im Bezug auf die Brewsterfenster nur p-polarisiertes Licht erzeugt, sollte in diesem
Versuchsaufbau vertikal polarisiertes Licht emittiert werden und somit ein Intensitätsmaximum bei $\varphi = \qty{0}{\degree}$ und
$\varphi = \qty{180}{\degree}$ vorhanden sein.
Es ist also ein Zusammenhang der Form
\begin{equation}
  \notag
  I(\varphi) = I_0 \sin^2\!\left(\varphi-\varphi_0\right)
\end{equation}
anzunehmen. Die Ausgleichsrechnung liefert hier für die Fitparameter die Werte
\begin{align*}
  I_0 &=\qty{3.97643+-0.00020}{\milli\watt} \\
 \varphi_0 &=\qty{-3.4549+-0.0006}{\degree}.
\end{align*}
In \autoref{fig:polarisation} ist dieser Fit ebenfalls eingezeichnet.
\begin{figure}[H]
  \centering
  \includegraphics{plot_pol.pdf}
  \caption{Intensität des Lasers in Abhängigkeit des Winkels des Polarisationsfilters.}
  \label{fig:polarisation}
\end{figure}

\subsection{Frequenzbreite}
Die Vermessung der Schwebungsfrequenz liefert für unterschiedliche Resonaterlängen mehrere PFrequenzpeaks im Oszilloskop, diese
sind in \autoref{tab:frequenz} festgehalten. Die Schwebungsfrequenz ist der mittlere Abstand zwischen den
Peaks und ist ebenfalls in \autoref{tab:frequenz} eingetragen.
Die Lichtverstärkung des He-Ne Lasers kann innerhalb eines Bereichs von $f'=\qty{1500}{\mega\hertz}$ (Doppler-Breite) stattfinden.
Die Anzahl der Möglichen Moden im Laser ergibt sich also daraus, wie oft die Schwebungsfrequenz in die Doppler-Breite passt.
Diese Überlegung ist in der letzten Spalte von \autoref{tab:frequenz} für alle Resonaterlängen zu finden.

\begin{table}[H]
  \centering
  \begin{tabular}{c|c|c|c}
    {$L$ / $\unit{\centi\metre}$} & {$f$ / $\unit{\mega\hertz}$} & {$\Delta f$ / $\unit{\mega\hertz}$} & {mod $\frac{f'}{\Delta f}$} \\
    \hline
    50  &  300, 600, 900 & 300 & 5\\
    75  &  203, 405, 604, 806 & $\approx$200 & 7 \\
    100  & 154, 304, 454, 608, 758 & $\approx$150 & 10 \\
    125  & 120, 240, 360, 480, 600, 720 & 120 & 12,5 \\
    150  & 101, 199, 300, 401, 500, 600 & $\approx$100 & 15 \\
    175  & 86, 173, 260, 345, 435, 518 & $\approx$85 & 17\\
    200  & 75, 154, 325, 375, 450 & 75 & 20
  \end{tabular}
  \caption{Frequenzpeaks in Abhängigkeit der Resonatorlänge für die erste Spiegelkonfiguration ($r_1=r_2=\qty{1400}{\milli\metre}$).}
  \label{tab:frequenz}
\end{table}
Es ist zu beobachten, dass die Schwebungsfrequenz $\Delta f$ mit zunehmender Resonaterlänge abnimmt. Dies ist damit zu begründen,
dass die benötigte Zeit der Photonenbunches zwischen den Spiegeln hin und her zu laufen proportional zum Spiegelabstand zunimmt.
Da die Frequenz antiproportional zur Zeit ist, nimmt diese somit ab.
Da alle gemessen Frequenzen wesentlich kleiner als die Doppler-Breite sind, läuft der Laser im Multimodenbetrieb. Erst durch hinreichend
kleine Resonaterlängen ließe sich ein Singlemode Betrieb einstellen.
Über den Zusammenhang
\begin{equation}
  \notag
  \Delta f = \frac{1}{T} = \frac{c}{2L}
\end{equation}
kann durch Umstellen die Resonatorlänge berechnet werden. Ein Vergleich mit der tatsächlich eingestellten Länge
ermöglicht dann eine Verifizierung des Messverfahrens.
Die berechneten Resonatorlängen sowie ihre Abweichung zu den tatsächlichen Längen sind in \autoref{tab:l_freq}
notiert.
\begin{table}[H]
  \centering
  \begin{tabular}{c|c|c}
    {$L$ / $\unit{\centi\metre}$} & {$L_{\text{exp}}$ / $\unit{\centi\metre}$} & {$\Delta L$ / $\unit{\percent}$} \\
    \hline
    50  &  49,97 & 0,06\\
    75  &  74,95 & 0,07 \\
    100  & 99,93 & 0,07  \\
    125  & 124,91 & 0,07  \\
    150  & 150,90 & 0,06 \\
    175  & 176,39 & 0,79 \\
    200  & 199,86 & 0,07
  \end{tabular}
  \caption{Frequenzpeaks in Abhängigkeit der Resonatorlänge für die erste Spiegelkonfiguration ($r_1=r_2=\qty{1400}{\milli\metre}$).}
  \label{tab:l_freq}
\end{table}
Allgemein lässt sich hier eine gute Übereinstimmung der Längen beobachten.

\subsection{Wellenlänge}
Die Wellenlänge des Lasers lässt sich mit Kenntniss der Abstände der Intensitätsmaxima einfach durch
Umstellen der Gleichung \eqref{eq:winkel} berechnen zu
\begin{equation}
  \label{eq:lambda}
  \lambda = \frac{g^{-1}}{n} \sin\left(\tan^{-1}\left(\frac{s_n}{d}\right)\right).
\end{equation}
Die Wellenlängen werden für die vier verschiedenen Gitter zuerst einzeln berechnet.
Im folgenden werden die Messwerte sowie daraus berechnete Wellenlängen in \autoref{tab:lambda1}, \autoref{tab:lambda2},
\autoref{tab:lambda3} und \autoref{tab:lambda4} dargestellt.
Dabei beschreibt $s_n$ den absolouten Abstand zwischen zwei Intensitätsmaxima und wird deshalb vor dem Einsetzen in
\eqref{eq:lambda} halbiert.
\begin{table}[H]
  \centering
  \begin{tabular}{c|c|c}
    $n$ & $s_n$ / $\unit{\centi\metre}$& $\lambda$ / $\unit{\nano\metre}$ \\
    \hline
    1 & 6,1 & 634.60 \\
    2 & 12,3 & 637.29 \\
    3 & 18,7 & 641.56 \\
    4 & 24,9 & 634.91
  \end{tabular}
  \caption{Messwerte und berechnete Wellenlängen für das erste Gitter ($g=\qty{80}{\per\milli\metre}$, $d=\qty{60}{\centi\metre}$).}
\label{tab:lambda1}
\end{table}
\begin{table}[H]
  \centering
  \begin{tabular}{c|c|c}
    $n$ & $s_n$ / $\unit{\centi\metre}$& $\lambda$ / $\unit{\nano\metre}$ \\
    \hline
    1 & 7,8 & 648.63 \\
    2 & 15,6 & 644.58 \\
    3 & 23,5 & 640.61 \\
    4 & 31,5 & 634.75
  \end{tabular}
  \caption{Messwerte und berechnete Wellenlängen für das zweite Gitter ($g=\qty{100}{\per\milli\metre}$, $d=\qty{60}{\centi\metre}$).}
\label{tab:lambda2}
\end{table}
\begin{table}[H]
  \centering
  \begin{tabular}{c|c|c}
    $n$ & $s_n$ / $\unit{\centi\metre}$& $\lambda$ / $\unit{\nano\metre}$ \\
    \hline
    1 & 25 & 641.03 \\
    2 & 72 & 640.18
  \end{tabular}
  \caption{Messwerte und berechnete Wellenlängen für das dritte Gitter ($g=\qty{600}{\per\milli\metre}$, $d=\qty{30}{\centi\metre}$).}
\label{tab:lambda3}
\end{table}
\begin{table}[H]
  \centering
  \begin{tabular}{c|c|c}
    $n$ & $s_n$ / $\unit{\centi\metre}$& $\lambda$ / $\unit{\nano\metre}$ \\
    \hline
    1 & 70 & 632.71 \\
  \end{tabular}
  \caption{Messwerte und berechnete Wellenlängen für das vierte Gitter ($g=\qty{1200}{\per\milli\metre}$, $d=\qty{30}{\centi\metre}$).}
\label{tab:lambda4}
\end{table}
Abschließend wird aus allen Wellenlängen der Mittelwert berechnet, das Ergebnis lautet
\begin{align*}
  \lambda &=\qty{639+-5}{\nano\metre}.
\end{align*}