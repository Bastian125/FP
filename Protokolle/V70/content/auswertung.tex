\section{Auswertung}
\label{sec:Auswertung}

\subsection{Turbomolekularpumpe}
\subsubsection{Leckratenmessung}
Es wird für die Leckratenmessung der Turbomolekularpumpe eine Messreihe aufgenommen. Die Messwerte sind in
\autoref{tab:turboleck} zu finden. In \autoref{fig:turboleck} ist der Druck $p$ in $\si{\milli\bar}$ gegen die
Zeit $t$ in $\si{\second}$ aufgetragen. Der Gleichgewichtsdruck ist $p_G \approx 6 \cdot 10^{-5} \,\si{\milli\bar}$.
\begin{table}[H]
  \centering
  \caption{Leckratenmessung der Turbomolekularpumpe}
  \label{tab:turboleck}
  \begin{tabular}{c c}
    \toprule
    $t/\si{\second}$ & $p/\si{\milli\bar}$ \\
    \midrule
    $  0$ & $(6,00 \pm 1,80) \cdot 10^{-5}$ \\
    $  5$ & $(9,93 \pm 2,98) \cdot 10^{-5}$ \\
    $ 10$ & $(1,22 \pm 0,37) \cdot 10^{-4}$ \\
    $ 15$ & $(1,50 \pm 0,45) \cdot 10^{-4}$ \\
    $ 20$ & $(1,79 \pm 0,54) \cdot 10^{-4}$ \\
    $ 25$ & $(2,06 \pm 0,62) \cdot 10^{-4}$ \\
    $ 30$ & $(2,33 \pm 0,70) \cdot 10^{-4}$ \\
    $ 35$ & $(2,59 \pm 0,78) \cdot 10^{-4}$ \\
    $ 40$ & $(2,84 \pm 0,85) \cdot 10^{-4}$ \\
    $ 45$ & $(3,11 \pm 0,93) \cdot 10^{-4}$ \\
    $ 50$ & $(3,36 \pm 1,01) \cdot 10^{-4}$ \\
    $ 55$ & $(3,64 \pm 1,09) \cdot 10^{-4}$ \\
    $ 60$ & $(3,92 \pm 1,18) \cdot 10^{-4}$ \\
    $ 65$ & $(4,15 \pm 1,24) \cdot 10^{-4}$ \\
    $ 70$ & $(4,42 \pm 1,33) \cdot 10^{-4}$ \\
    $ 75$ & $(4,69 \pm 1,41) \cdot 10^{-4}$ \\
    $ 80$ & $(4,95 \pm 1,49) \cdot 10^{-4}$ \\
    $ 85$ & $(5,20 \pm 1,56) \cdot 10^{-4}$ \\
    $ 90$ & $(5,47 \pm 1,64) \cdot 10^{-4}$ \\
    $ 95$ & $(5,73 \pm 1,72) \cdot 10^{-4}$ \\
    $100$ & $(5,99 \pm 1,80) \cdot 10^{-4}$ \\
    $105$ & $(6,23 \pm 1,87) \cdot 10^{-4}$ \\
    $110$ & $(6,50 \pm 1,95) \cdot 10^{-4}$ \\
    $115$ & $(6,76 \pm 2,03) \cdot 10^{-4}$ \\
    $120$ & $(7,17 \pm 2,15) \cdot 10^{-4}$ \\
    \bottomrule
  \end{tabular}
\end{table}

\begin{figure}[H]
  \centering
  \includegraphics[width=0.8\textwidth]{build/turboleck.pdf}
  \caption{Grafische Darstellung der Messwerte für die Leckratenmessung der Tubromolekularpumpe.}
  \label{fig:turboleck}
\end{figure}
Außerdem wurde eine lineare Regression der Form $f(x) = ax +b$ durchgeführt. Dadurch lässt sich das Saugvermögen
$S$ durch $S = \frac{V}{p_G}\cdot a$ bestimmen. Die Turbomolekularpumpe hat dabei ein Volumen von
$V = (33 \pm 0,1) \,\si{\liter}$. 
\begin{align*}
  a &= (5,30 \pm 0,1)\cdot 10^{-5} \si{\milli\bar\per\second} \\
  p_G &= (6,00 \pm 1,80) \cdot 10^{-5} \si{\milli\bar} \\
  S &= (29 \pm 9)\,\si{\liter\per\second}
\end{align*}
bestimmen.

\subsubsection{Evakuierungsmessung}
Es wurden für den Startdruck $p_{0} \approx 5\cdot 10^{-3}\,\si{\milli\bar}$ drei Messreihen aufgenommen und
gemittelt. Für die weitere Auswertung wurde noch $ln(\frac{p(t)-p_0}{p_0 - p_E})$ bestimmt und gegen die Zeit
$t$ in $\si{\second}$ aufgetragen. Die entsprechenden Werte, der Enddruck $p_{\symup{E}}$ und die dazu passende
grafische Abbildung sind in \autoref{tab:turboevak} und \autoref{fig:turboevak} zu finden.

\begin{table}[H]
  \centering
  \caption{Messdaten und daraus berechnete Größen der Evakuierungsmessung.}
  \label{tab:turboevak}
  \begin{tabular}{c c c c c }
    \toprule
    $t/s$ & $p_1 /\si{\milli\bar}$ & $p_2 /\si{\milli\bar}$ & $p_3 /\si{\milli\bar}$ & $p_m /\si{\milli\bar}$ \\
    \midrule
    $  0$ & $ (5,02 \pm 1,51)\cdot 10^{-3}$ & $ (5,17 \pm 1,55)\cdot 10^{-3}$ & $ (5,12 \pm 1,54)\cdot 10^{-3}$ & $(5,10 \pm 0,06)\cdot 10^{-3} $ \\
    $  5$ & $ (4,93 \pm 1,48)\cdot 10^{-4}$ & $ (4,78 \pm 1,43)\cdot 10^{-4}$ & $ (5,15 \pm 1,55)\cdot 10^{-4}$ & $(4,95 \pm 0,15)\cdot 10^{-4} $ \\
    $ 10$ & $ (1,39 \pm 0,42)\cdot 10^{-4}$ & $ (1,32 \pm 0,40)\cdot 10^{-4}$ & $ (1,49 \pm 0,45)\cdot 10^{-4}$ & $(1,40 \pm 0,07)\cdot 10^{-4} $ \\
    $ 15$ & $ (6,01 \pm 1,80)\cdot 10^{-5}$ & $ (6,05 \pm 1,82)\cdot 10^{-5}$ & $ (6,52 \pm 1,96)\cdot 10^{-5}$ & $(6,19 \pm 0,23)\cdot 10^{-5} $ \\
    $ 20$ & $ (3,65 \pm 1,10)\cdot 10^{-5}$ & $ (3,78 \pm 1,13)\cdot 10^{-5}$ & $ (4,16 \pm 1,25)\cdot 10^{-5}$ & $(3,86 \pm 0,22)\cdot 10^{-5} $ \\
    $ 25$ & $ (3,04 \pm 0,91)\cdot 10^{-5}$ & $ (3,04 \pm 0,91)\cdot 10^{-5}$ & $ (3,34 \pm 1,00)\cdot 10^{-5}$ & $(3,14 \pm 0,14)\cdot 10^{-5} $ \\
    $ 30$ & $ (2,75 \pm 0,83)\cdot 10^{-5}$ & $ (3,17 \pm 0,95)\cdot 10^{-5}$ & $ (3,02 \pm 0,91)\cdot 10^{-5}$ & $(2,98 \pm 0,17)\cdot 10^{-5} $ \\
    $ 35$ & $ (2,56 \pm 0,77)\cdot 10^{-5}$ & $ (2,86 \pm 0,86)\cdot 10^{-5}$ & $ (2,80 \pm 0,84)\cdot 10^{-5}$ & $(2,74 \pm 0,13)\cdot 10^{-5} $ \\
    $ 40$ & $ (2,44 \pm 0,73)\cdot 10^{-5}$ & $ (2,67 \pm 0,80)\cdot 10^{-5}$ & $ (2,63 \pm 0,79)\cdot 10^{-5}$ & $(2,58 \pm 0,10)\cdot 10^{-5} $ \\
    $ 45$ & $ (2,34 \pm 0,70)\cdot 10^{-5}$ & $ (2,52 \pm 0,76)\cdot 10^{-5}$ & $ (2,50 \pm 0,75)\cdot 10^{-5}$ & $(2,45 \pm 0,08)\cdot 10^{-5} $ \\
    $ 50$ & $ (2,26 \pm 0,68)\cdot 10^{-5}$ & $ (2,32 \pm 0,70)\cdot 10^{-5}$ & $ (2,38 \pm 0,71)\cdot 10^{-5}$ & $(2,32 \pm 0,05)\cdot 10^{-5} $ \\
    $ 55$ & $ (2,19 \pm 0,66)\cdot 10^{-5}$ & $ (2,24 \pm 0,67)\cdot 10^{-5}$ & $ (2,29 \pm 0,69)\cdot 10^{-5}$ & $(2,24 \pm 0,04)\cdot 10^{-5} $ \\
    $ 60$ & $ (2,13 \pm 0,64)\cdot 10^{-5}$ & $ (2,18 \pm 0,65)\cdot 10^{-5}$ & $ (2,22 \pm 0,67)\cdot 10^{-5}$ & $(2,18 \pm 0,04)\cdot 10^{-5} $ \\
    $ 65$ & $ (2,08 \pm 0,62)\cdot 10^{-5}$ & $ (2,12 \pm 0,64)\cdot 10^{-5}$ & $ (2,15 \pm 0,65)\cdot 10^{-5}$ & $(2,12 \pm 0,03)\cdot 10^{-5} $ \\
    $ 70$ & $ (2,04 \pm 0,61)\cdot 10^{-5}$ & $ (2,07 \pm 0,62)\cdot 10^{-5}$ & $ (2,10 \pm 0,63)\cdot 10^{-5}$ & $(2,07 \pm 0,02)\cdot 10^{-5} $ \\
    $ 75$ & $ (1,99 \pm 0,60)\cdot 10^{-5}$ & $ (2,02 \pm 0,61)\cdot 10^{-5}$ & $ (2,05 \pm 0,62)\cdot 10^{-5}$ & $(2,02 \pm 0,02)\cdot 10^{-5} $ \\
    $ 80$ & $ (1,96 \pm 0,59)\cdot 10^{-5}$ & $ (1,98 \pm 0,59)\cdot 10^{-5}$ & $ (2,00 \pm 0,60)\cdot 10^{-5}$ & $(1,98 \pm 0,02)\cdot 10^{-5} $ \\
    $ 85$ & $ (1,92 \pm 0,58)\cdot 10^{-5}$ & $ (1,95 \pm 0,59)\cdot 10^{-5}$ & $ (1,96 \pm 0,59)\cdot 10^{-5}$ & $(1,94 \pm 0,02)\cdot 10^{-5} $ \\
    $ 90$ & $ (1,90 \pm 0,57)\cdot 10^{-5}$ & $ (1,91 \pm 0,57)\cdot 10^{-5}$ & $ (1,93 \pm 0,58)\cdot 10^{-5}$ & $(1,91 \pm 0,01)\cdot 10^{-5} $ \\
    $ 95$ & $ (1,87 \pm 0,56)\cdot 10^{-5}$ & $ (1,89 \pm 0,57)\cdot 10^{-5}$ & $ (1,90 \pm 0,57)\cdot 10^{-5}$ & $(1,89 \pm 0,01)\cdot 10^{-5} $ \\
    $100$ & $ (1,85 \pm 0,56)\cdot 10^{-5}$ & $ (1,86 \pm 0,56)\cdot 10^{-5}$ & $ (1,87 \pm 0,56)\cdot 10^{-5}$ & $(1,86 \pm 0,01)\cdot 10^{-5} $ \\
    $105$ & $ (1,82 \pm 0,55)\cdot 10^{-5}$ & $ (1,83 \pm 0,55)\cdot 10^{-5}$ & $ (1,84 \pm 0,55)\cdot 10^{-5}$ & $(1,83 \pm 0,01)\cdot 10^{-5} $ \\
    $110$ & $ (1,80 \pm 0,54)\cdot 10^{-5}$ & $ (1,81 \pm 0,54)\cdot 10^{-5}$ & $ (1,81 \pm 0,54)\cdot 10^{-5}$ & $(1,81 \pm 0,01)\cdot 10^{-5} $ \\
    $115$ & $ (1,78 \pm 0,53)\cdot 10^{-5}$ & $ (1,79 \pm 0,54)\cdot 10^{-5}$ & $ (1,79 \pm 0,54)\cdot 10^{-5}$ & $(1,79 \pm 0,01)\cdot 10^{-5} $ \\
    $120$ & $ (1,76 \pm 0,53)\cdot 10^{-5}$ & $ (1,77 \pm 0,53)\cdot 10^{-5}$ & $ (1,77 \pm 0,53)\cdot 10^{-5}$ & $(1,77 \pm 0,01)\cdot 10^{-5} $ \\
    \bottomrule
  \end{tabular}
\end{table}

\begin{figure}[H]
  \centering
  \includegraphics[width=0.8\textwidth]{build/turboevak.pdf}
  \caption{Grafische Darstellung der Messwerte für die Evakuierungsmessungnmessung der Tubromolekularpumpe.}
  \label{fig:turboevak}
\end{figure}

Es wurden für zwei Bereiche jeweils lineare Ausgleichsrechnungen der Form $f(x)=ax+b$ durchgeführt und deren
Parameter bestimmt. Aus $S=-aV$ lässt sich das Saugvermögen der Pumpe bestimmen. Für den ersten Bereich ergibt
sich
\begin{align*}
  a &= (-0,312 \pm 0,044)\,\si{\per\second}\\
  S &= (10,3 \pm 1,5)\,\si{\liter\per\second}.
\end{align*}
Und für den zweiten Bereich
\begin{align*}
  a &= (-0,038 \pm 0,001)\,\si{\per\second}\\
  S &=(1,254 \pm 0,033)\,\si{\liter\per\second}.
\end{align*}

\subsection{Drehschieberpumpe}
\subsubsection{Leckratenmessung}
Für die Leckratenmessung wurden Messwerte für vier Gleichgewichtsdrücke aufgenommen. Für den ersten
Gleichgewichtsdruck von $p_{1, 2, 3} \approx \SI{1,0}{\milli\bar}$ wurde dreimal gemessen.
Die Messwerte und der daraus bestimmte Mittelwert sind in \autoref{tab:pm} zu finden. Der entsprechende Plot
ist in \autoref{fig:pm} dargestellt.

\begin{table}[H]
  \centering
  \caption{Messwerte für den ersten Gleichgewichtsdruck und der daraus resultierende Mittelwert.}
  \label{tab:pm}
  \begin{tabular}{c c c c c}
    \toprule
  $t/\si{\second}$ & $p_1 /\si{\milli\bar}$ & $p_2 /\si{\milli\bar}$ & $p_3 /\si{\milli\bar}$ & $p_m /\si{\milli\bar}$ \\
    \midrule
    $  0$ & $ 1,0 \pm 0,1 $ & $ 1,0 \pm 0,1 $ & $ 1,0 \pm 0,1 $& $ 1,0 \pm 0,00 $ \\ 
    $ 10$ & $ 3,3 \pm 0,3 $ & $ 3,5 \pm 0,4 $ & $ 3,2 \pm 0,3 $& $ 3,3 \pm 0,12 $ \\ 
    $ 20$ & $ 3,7 \pm 0,4 $ & $ 3,9 \pm 0,4 $ & $ 3,6 \pm 0,4 $& $ 3,7 \pm 0,12 $ \\ 
    $ 30$ & $ 4,0 \pm 0,4 $ & $ 4,1 \pm 0,4 $ & $ 3,9 \pm 0,4 $& $ 4,0 \pm 0,08 $ \\ 
    $ 40$ & $ 4,3 \pm 0,4 $ & $ 4,4 \pm 0,4 $ & $ 4,2 \pm 0,4 $& $ 4,3 \pm 0,08 $ \\ 
    $ 50$ & $ 4,5 \pm 0,5 $ & $ 4,6 \pm 0,5 $ & $ 4,5 \pm 0,5 $& $ 4,5 \pm 0,05 $ \\ 
    $ 60$ & $ 4,8 \pm 0,5 $ & $ 5,0 \pm 0,5 $ & $ 4,8 \pm 0,5 $& $ 4,9 \pm 0,09 $ \\ 
    $ 70$ & $ 5,2 \pm 0,5 $ & $ 5,4 \pm 0,5 $ & $ 5,2 \pm 0,5 $& $ 5,3 \pm 0,09 $ \\ 
    $ 80$ & $ 5,5 \pm 0,6 $ & $ 5,7 \pm 0,6 $ & $ 5,5 \pm 0,6 $& $ 5,6 \pm 0,09 $ \\ 
    $ 90$ & $ 5,8 \pm 0,6 $ & $ 6,0 \pm 0,6 $ & $ 5,8 \pm 0,6 $& $ 5,9 \pm 0,09 $ \\ 
    $100$ & $ 6,1 \pm 0,6 $ & $ 6,2 \pm 0,6 $ & $ 6,1 \pm 0,6 $& $ 6,1 \pm 0,05 $ \\ 
    $110$ & $ 6,4 \pm 0,6 $ & $ 6,5 \pm 0,7 $ & $ 6,4 \pm 0,6 $& $ 6,4 \pm 0,05 $ \\ 
    $120$ & $ 6,6 \pm 0,7 $ & $ 6,7 \pm 0,7 $ & $ 6,6 \pm 0,7 $& $ 6,6 \pm 0,05 $ \\ 
    $130$ & $ 6,9 \pm 0,7 $ & $ 7,0 \pm 0,7 $ & $ 6,9 \pm 0,7 $& $ 6,9 \pm 0,05 $ \\ 
    $140$ & $ 7,2 \pm 0,7 $ & $ 7,3 \pm 0,7 $ & $ 7,2 \pm 0,7 $& $ 7,2 \pm 0,05 $ \\ 
    $150$ & $ 7,4 \pm 0,7 $ & $ 7,6 \pm 0,8 $ & $ 7,5 \pm 0,8 $& $ 7,5 \pm 0,08 $ \\ 
    $160$ & $ 7,7 \pm 0,8 $ & $ 7,8 \pm 0,8 $ & $ 7,7 \pm 0,8 $& $ 7,7 \pm 0,05 $ \\ 
    $170$ & $ 7,9 \pm 0,8 $ & $ 8,0 \pm 0,8 $ & $ 8,0 \pm 0,8 $& $ 8,0 \pm 0,09 $ \\ 
    $180$ & $ 8,1 \pm 0,8 $ & $ 8,3 \pm 0,8 $ & $ 8,1 \pm 0,8 $& $ 8,2 \pm 0,09 $ \\ 
    $190$ & $ 8,4 \pm 0,8 $ & $ 8,5 \pm 0,9 $ & $ 8,4 \pm 0,8 $& $ 8,4 \pm 0,05 $ \\ 
    $200$ & $ 8,5 \pm 0,9 $ & $ 8,7 \pm 0,9 $ & $ 8,6 \pm 0,9 $& $ 8,6 \pm 0,08 $ \\ 
    \bottomrule
  \end{tabular}
\end{table}

\begin{figure}[H]
  \centering
  \includegraphics[width=0.8\textwidth]{build/pm.pdf}
  \caption{Grafische Darstellung der Messwerte für die Leckratenmessung der Drehschieberpumpe für den
  Gleichgewichtsdruck $p_{\symup{G}} = (1,0 \pm 0,1)\si{\milli\bar}$.}
  \label{fig:pm}
\end{figure}

Die Messwerte der anderen drei Gleichgewichtsdrücke sind in \autoref{tab:drehleck} dargestellt. Die entsprechenden
Plots sind in \autoref{fig:drehleck1}, \autoref{fig:drehleck2} und \autoref{fig:drehleck3} zu finden.
Die weiteren Gleichgewichtsdrücke sind
$p_4 \approx 10 \cdot \,\si{\milli\bar}$, $p_5 \approx 50 \cdot \,\si{\milli\bar}$ und
$p_6 \approx 100 \cdot \,\si{\milli\bar}$.

\begin{table}[H]
  \centering
  \caption{Messwerte für die drei anderen Gleichgewichtsdrücke.}
  \label{tab:drehleck}
  \begin{tabular}{c c c c}
    \toprule
    $t/\si{\second}$ & $p_4 /\si{\milli\bar}$ & $p_5 /\si{\milli\bar}$ & $p_6 /\si{\milli\bar}$ \\
    \midrule
    $  0$ & $ 10,0 \pm 3,6 $ & $  50,0 \pm  3,6 $ & $ 100,0 \pm 3,6 $ \\ 
    $ 10$ & $ 18,7 \pm 3,6 $ & $  77,3 \pm  3,6 $ & $ 144,4 \pm 3,6 $ \\ 
    $ 20$ & $ 23,0 \pm 3,6 $ & $  95,7 \pm  3,6 $ & $ 187,8 \pm 3,6 $ \\ 
    $ 30$ & $ 27,2 \pm 3,6 $ & $ 114,2 \pm  3,6 $ & $ 211,7 \pm 3,6 $ \\ 
    $ 40$ & $ 31,0 \pm 3,6 $ & $ 132,5 \pm  3,6 $ & $ 245,9 \pm 3,6 $ \\ 
    $ 50$ & $ 35,8 \pm 3,6 $ & $ 150,9 \pm  3,6 $ & $ 280,1 \pm 3,6 $ \\ 
    $ 60$ & $ 40,0 \pm 3,6 $ & $ 169,3 \pm  3,6 $ & $ 314,4 \pm 3,6 $ \\ 
    $ 70$ & $ 44,8 \pm 3,6 $ & $ 187,7 \pm  3,6 $ & $ 348,5 \pm 3,6 $ \\ 
    $ 80$ & $ 48,6 \pm 3,6 $ & $ 205,2 \pm  3,6 $ & $ 382,6 \pm 3,6 $ \\ 
    $ 90$ & $ 53,0 \pm 3,6 $ & $ 223,6 \pm  3,6 $ & $ 416,9 \pm 3,6 $ \\ 
    $100$ & $ 57,2 \pm 3,6 $ & $ 242,1 \pm  3,6 $ & $ 450,0 \pm 3,6 $ \\ 
    $110$ & $ 61,5 \pm 3,6 $ & $ 260,3 \pm  3,6 $ & $ 483,6 \pm 3,6 $ \\ 
    $120$ & $ 66,1 \pm 3,6 $ & $ 280,7 \pm  3,6 $ & $ 516,4 \pm 3,6 $ \\ 
    $130$ & $ 70,4 \pm 3,6 $ & $ 300,9 \pm  3,6 $ & $ 549,9 \pm 3,6 $ \\ 
    $140$ & $ 75,2 \pm 3,6 $ & $ 315,5 \pm  3,6 $ & $ 581,2 \pm 3,6 $ \\ 
    $150$ & $ 78,4 \pm 3,6 $ & $ 334,1 \pm 13,6 $ & $ 612,7 \pm 3,6 $ \\ 
    $160$ & $ 82,8 \pm 3,6 $ & $ 352,3 \pm 13,6 $ & $ 643,4 \pm 3,6 $ \\ 
    $170$ & $ 87,1 \pm 3,6 $ & $ 370,8 \pm 13,6 $ & $ 673,1 \pm 3,6 $ \\ 
    $180$ & $ 91,3 \pm 3,6 $ & $ 389,9 \pm 13,6 $ & $ 702,0 \pm 3,6 $ \\ 
    $190$ & $ 95,5 \pm 3,6 $ & $ 407,2 \pm 13,6 $ & $ 729,9 \pm 3,6 $ \\ 
    $200$ & $ 99,8 \pm 3,6 $ & $ 425,6 \pm 13,6 $ & $ 756,5 \pm 3,6 $ \\ 
    \bottomrule
  \end{tabular}
\end{table}

\begin{figure}[H]
  \centering
  \includegraphics[width=0.8\textwidth]{build/p4.pdf}
  \caption{Grafische Darstellung der Messwerte für die Leckratenmessung der Drehschieberpumpe mit dem
  Gleichgewichtsdruck $p_4 \approx (10 \pm 3,6)\, \si{\milli\bar}$.}
  \label{fig:drehleck1}
\end{figure}

\begin{figure}[H]
  \centering
  \includegraphics[width=0.8\textwidth]{build/p5.pdf}
  \caption{Grafische Darstellung der Messwerte für die Leckratenmessung der Drehschieberpumpe mit dem
  Gleichgewichtsdruck $p_5 \approx (50 \pm 3,6)\,\si{\milli\bar}$.}
  \label{fig:drehleck2}
\end{figure}

\begin{figure}[H]
  \centering
  \includegraphics[width=0.8\textwidth]{build/p6.pdf}
  \caption{Grafische Darstellung der Messwerte für die Leckratenmessung der Drehschieberpumpe mit dem
  Gleichgewichtsdruck $p_6 \approx (100 \pm 3,6) \,\si{\milli\bar}$.}
  \label{fig:drehleck3}
\end{figure}
Analog zur Turbomolekularpumpe wurden wieder lineare Ausgleichsrechnungen durchgeführt und das Saugvermögen bestimmt.
Der einzige Unterschied besteht darin, dass das Volumen nun $V=(34 \pm 0.1)\,\si{\liter}$ beträgt.
Für den gemittelten Druck ergibts sich
\begin{align*}
  a &= (0,03 \pm 0,002)\cdot \si{\milli\bar\per\second} \\
  p_G &= (1,00 \pm 0,1)  \si{\milli\bar} \\
  S &= (1,02 \pm 0,12)\,\si{\liter\per\second}.
\end{align*}

Für $p_4$ ergibt sich
\begin{align*}
  a &= (0,43 \pm 0,003)\cdot \si{\milli\bar\per\second} \\
  p_G &= (10,0 \pm 3,6)  \si{\milli\bar} \\
  S &= (1,5 \pm 0,5)\,\si{\liter\per\second}.
\end{align*}

Für $p_5$ ergibt sich
\begin{align*}
  a &= (1,85 \pm 0,008)\cdot \si{\milli\bar\per\second} \\
  p_G &= (50,0 \pm 3,6)  \si{\milli\bar} \\
  S &= (1,26 \pm 0,1)\,\si{\liter\per\second}.
\end{align*}

Für $p_6$ ergibt sich
\begin{align*}
  a &= (3,28 \pm 0,026) \si{\milli\bar\per\second} \\
  p_G &= (100,0 \pm 3,6) \si{\milli\bar} \\
  S &= (1,12 \pm 0,1)\,\si{\liter\per\second}.
\end{align*}


\subsubsection{Evakuierungsmessung}
Die Messwerte der Evakuierungsmessung der Drehschieberpumpe sind in \autoref{tab:drehevak} dargestellt. In
\autoref{fig:drehevak} ist $ln(\frac{p(t)-p_0}{p_0 - p_E})$ gegen die Zeit $t$ in $\si{\second}$ aufgetragen. Der
Startdruck $p_{0}$ und der Enddruck $p_{\symup{E}}$ sind dabei jeweils der erste und letzte Wert in
\autoref{tab:drehevak}.

\begin{table}[H]
  \centering
  \caption{Messwerte der Evakuierungsmessung der Drehschieberpumpe.}
  \label{tab:drehevak}
  \begin{tabular}{c c c | c c c}
    \toprule
    $t/\si{\second}$ & $p/\si{\milli\bar}$ & $ln(\frac{p(t)-p_0}{p_0 - p_E})$ & $t/\si{\second}$ & $p/\si{\milli\bar}$ & $ln(\frac{p(t)-p_0}{p_0 - p_E})$\\ 
    \midrule
    $  0$ & $997,1 \pm  3,6$  &                  & $310$ & $ 0,60 \pm 0,06 $ & $ -7,64 \pm 0,01$ \\ 
    $ 10$ & $649,1 \pm  3,6$  & $ 0,00         $ & $320$ & $ 0,55 \pm 0,06 $ & $ -7,70 \pm 0,14$ \\
    $ 20$ & $479,6 \pm  3,6$  & $-0,43 \pm 0,01$ & $330$ & $ 0,52 \pm 0,05 $ & $ -7,82 \pm 0,15$ \\
    $ 30$ & $355,2 \pm  3,6$  & $-0,73 \pm 0,01$ & $340$ & $ 0,47 \pm 0,05 $ & $ -7,90 \pm 0,14$ \\
    $ 40$ & $263,7 \pm  3,6$  & $-1,03 \pm 0,01$ & $350$ & $ 0,45 \pm 0,05 $ & $ -8,04 \pm 0,16$ \\
    $ 50$ & $190,6 \pm  3,6$  & $-1,33 \pm 0,01$ & $360$ & $ 0,42 \pm 0,04 $ & $ -8,11 \pm 0,17$ \\
    $ 60$ & $139,0 \pm  3,6$  & $-1,66 \pm 0,02$ & $370$ & $ 0,40 \pm 0,04 $ & $ -8,21 \pm 0,15$ \\
    $ 70$ & $100,9 \pm  3,6$  & $-1,97 \pm 0,03$ & $380$ & $ 0,38 \pm 0,04 $ & $ -8,29 \pm 0,16$ \\
    $ 80$ & $ 72,5 \pm  3,6$  & $-2,29 \pm 0,04$ & $390$ & $ 0,36 \pm 0,04 $ & $ -8,37 \pm 0,18$ \\
    $ 90$ & $ 52,3 \pm  3,6$  & $-2,62 \pm 0,05$ & $400$ & $ 0,33 \pm 0,03 $ & $ -8,47 \pm 0,20$ \\
    $100$ & $ 36,9 \pm  3,6$  & $-2,95 \pm 0,07$ & $410$ & $ 0,32 \pm 0,03 $ & $ -8,62 \pm 0,18$ \\
    $110$ & $ 26,2 \pm  3,6$  & $-3,30 \pm 0,10$ & $420$ & $ 0,30 \pm 0,03 $ & $ -8,68 \pm 0,19$ \\
    $120$ & $ 18,2 \pm  3,6$  & $-3,64 \pm 0,14$ & $430$ & $ 0,29 \pm 0,03 $ & $ -8,80 \pm 0,21$ \\
    $130$ & $ 13,2 \pm  3,6$  & $-4,01 \pm 0,20$ & $440$ & $ 0,28 \pm 0,03 $ & $ -8,87 \pm 0,23$ \\
    $140$ & $ 10,0 \pm  3,6$  & $-4,34 \pm 0,28$ & $450$ & $ 0,27 \pm 0,03 $ & $ -8,94 \pm 0,24$ \\
    $150$ & $  7,7 \pm 0,77$  & $-4,62 \pm 0,37$ & $460$ & $ 0,26 \pm 0,03 $ & $ -9,02 \pm 0,26$ \\
    $160$ & $  6,0 \pm 0,60$  & $-4,88 \pm 0,10$ & $470$ & $ 0,25 \pm 0,03 $ & $ -9,11 \pm 0,29$ \\
    $170$ & $  4,4 \pm 0,44$  & $-5,14 \pm 0,10$ & $480$ & $ 0,24 \pm 0,02 $ & $ -9,21 \pm 0,32$ \\
    $180$ & $  3,5 \pm 0,35$  & $-5,46 \pm 0,10$ & $490$ & $ 0,23 \pm 0,02 $ & $ -9,31 \pm 0,25$ \\
    $190$ & $  2,8 \pm 0,28$  & $-5,70 \pm 0,10$ & $500$ & $ 0,22 \pm 0,02 $ & $ -9,43 \pm 0,28$ \\
    $200$ & $  2,2 \pm 0,22$  & $-5,93 \pm 0,11$ & $510$ & $ 0,21 \pm 0,02 $ & $ -9,56 \pm 0,32$ \\
    $210$ & $  1,9 \pm 0,19$  & $-6,19 \pm 0,11$ & $520$ & $ 0,20 \pm 0,02 $ & $ -9,72 \pm 0,37$ \\
    $220$ & $  1,6 \pm 0,16$  & $-6,35 \pm 0,11$ & $530$ & $ 0,19 \pm 0,02 $ & $ -9,90 \pm 0,45$ \\
    $230$ & $  1,4 \pm 0,14$  & $-6,53 \pm 0,11$ & $540$ & $ 0,19 \pm 0,02 $ & $-10,12 \pm 0,56$ \\
    $240$ & $  1,2 \pm 0,12$  & $-6,68 \pm 0,11$ & $550$ & $ 0,18 \pm 0,02 $ & $-10,12 \pm 0,56$ \\
    $250$ & $  1,0 \pm 0,10$  & $-6,86 \pm 0,11$ & $560$ & $ 0,17 \pm 0,02 $ & $-10,41 \pm 0,75$ \\
    $260$ & $  0,3 \pm 0,03$  & $-7,07 \pm 0,12$ & $570$ & $ 0,17 \pm 0,02 $ & $-10,82 \pm 1,12$ \\
    $270$ & $  0,4 \pm 0,04$  & $-7,15 \pm 0,01$ & $580$ & $ 0,16 \pm 0,02 $ & $-10,82 \pm 1,12$ \\
    $280$ & $  0,6 \pm 0,06$  & $-7,28 \pm 0,01$ & $590$ & $ 0,15 \pm 0,01 $ & $-11,51 \pm 2,24$ \\
    $290$ & $  0,0         $  & $-7,40 \pm 0,10$ & $600$ & $ 0,15 \pm 0,01 $ &                   \\
    $300$ & $  0,3 \pm 0,03$  & $-7,50 \pm 0,01$ &       &                   &                   \\
    \bottomrule
  \end{tabular}
\end{table}

\begin{figure}[H]
  \centering
  \includegraphics[width=0.8\textwidth]{build/drehevak.pdf}
  \caption{Grafische Darstellung der Messwerte für die Evakuierungsmessungnmessung der Drehschieberpumpe.}
  \label{fig:drehevak}
\end{figure}

Es wurden für zwei Bereiche jeweils lineare Ausgleichsrechnungen der Form $f(x)=ax+b$ durchgeführt und deren
Parameter bestimmt. Aus $S=-aV$ lässt sich das Saugvermögen der Pumpe bestimmen. Für den ersten Bereich ergibt
sich
\begin{align*}
  a &= (-0,031 \pm 0,001)\,\si{\per\second}\\
  S &= (1,054 \pm 0,034)\,\si{\liter\per\second}.
\end{align*}
Und für den zweiten Bereich
\begin{align*}
  a &= (-0,011 \pm 0,001)\,\si{\per\second}\\
  S &= (0,374 \pm 0,034)\,\si{\liter\per\second}.
\end{align*}