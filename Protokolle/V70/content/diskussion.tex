\section{Diskussion}
\label{sec:Diskussion}
Abschließend lässt sich fest stellen, dass die Messung vollständig nach den theoretisch hergeleiteten Verhalten
abliefen. Die bestimmten Saugvermögen und die relativen Abweicchungen sind in \autoref{tab:S1} für die
Turbomolekularpumpe und in \autoref{tab:S2} für die Drehschieberpumpe zu finden. Allgemein lässt sich feststellen,
dass die Leckratenmessungen für die Turbomolekularpumpe wesentlich bessere Ergebnis liefert.
Außerdem fällt auf, dass der zweite Bereich der Evakuierungsmessung für beide Pumpen signifikant schlechtere
Ergebnisse liefert. Insgesamt lässt sich sagen, dass die Ergebnisse die Turbomolekularpumpe hohe relative
Abweichungen zu den Herstellerangaben haben, so dass die Messung nicht vollständig zufriedendstellend ist.
Die Abweichungen lassen sich höchstwahrscheinlich auf den Einfluss des Leitwerts zurück führen. Im
Gegensatz zu den bestimmten Saugvermögen der Turbomolekularpumpe haben die bestimmten Saugvermögen der
Drehschieberpumpe geringe Abweichungen, so dass die Messung erfolgreich war. Auch hier sind die Ergebnisse der
Leckratenmessungen meist besser und es fällt auf, dass bei höheren Gleichgewichtsdrücken tendenziell bessere
Ergebnisse erzielt wurden. 
\begin{table}
    \centering
    \caption{Saugvermögen und deren realtiven Abweichungen zu den Herstellerangaben der Turbomolekularpumpe. Das
    Saugvermögen wird mit $S=77\,\si{\liter\per\second}$ angegeben.}
    \label{tab:S1}
    \begin{tabular}{c c c}
        \toprule
        Messung & $S/\,\si{\liter\per\second}$ & relative Abweichung in Prozent \\
        \midrule
        Evakuierung für den ersten Bereich & $S = (10,3 \pm 1,5)\,\si{\liter\per\second}$ & $86,6$ \\
        Evakuierung für den zweiten Bereich & $S =(1,254 \pm 0,033)\,\si{\liter\per\second}$ & $98,4 $ \\
        Leckrate mit $p_G \approx 6 \cdot 10^{-5} \,\si{\milli\bar}$ & $S = (29 \pm 9)\,\si{\liter\per\second}$ & $62,0$\\
        \bottomrule
    \end{tabular}
\end{table}

\begin{table}
    \centering
    \caption{Saugvermögen und deren realtiven Abweichungen zu den Herstellerangaben der Drehschieberpumpe. Das
    Saugvermögen wird mit $S=1,1\,\si{\liter\per\second}$ angegeben.}
    \label{tab:S2}
    \begin{tabular}{c c c}
        \toprule
        Messung & $S/\,\si{\liter\per\second}$ & relative Abweichung in Prozent \\
        \midrule
        Evakuierung für den ersten Bereich & $S = (1,254 \pm 0,014)\,\si{\liter\per\second}$ & $14,00$ \\
        Evakuierung für den zweiten Bereich & $S = (0,363 \pm 0,010)\,\si{\liter\per\second}$ & $67,00$ \\
        Leckrate mit $p_G \approx 1 \,\si{\milli\bar}$ & $S = (1,02 \pm 0,22)\,\si{\liter\per\second}$ & $7,72$\\
        Leckrate mit $p_G \approx 10 \,\si{\milli\bar}$ & $S = (1,5 \pm 0,4)\,\si{\liter\per\second}$ & $36,36$\\
        Leckrate mit $p_G \approx 50 \,\si{\milli\bar}$ & $S = (1,3 \pm 0,4)\,\si{\liter\per\second}$ & $18,18$\\
        Leckrate mit $p_G \approx 100 \,\si{\milli\bar}$ & $S = (1,12 \pm 0,33)\,\si{\liter\per\second}$ & $1,81$\\
        \bottomrule
    \end{tabular}
\end{table}